\documentclass[a4paper,10pt]{paper}

\usepackage[utf8]{inputenc}
\setlength{\parindent}{0pt}
%% say no to serif
\renewcommand*\familydefault{\sfdefault}
\usepackage{amsmath}
\usepackage[inner=2.5cm,outer=2.5cm,bottom=2.5cm,top=2.5cm]{geometry}

%% allow referencing of enumerated lists
\usepackage{enumitem}

%% bibliography
\usepackage{nameref}
\usepackage[british]{babel}
\usepackage[
    style=apa,
    backend=biber,
    url=true, 
    doi=false,
    eprint=false,
    bibencoding=utf8
]{biblatex}
\DeclareLanguageMapping{british}{british-apa}
\DeclareFieldFormat{apacase}{#1}  % otherwise apa sentence cases
\addbibresource{corpora.bib}

% ==========
% non-R code
% ==========

\RequirePackage{listings}
\lstset{
    basicstyle=\footnotesize,
    xleftmargin=1.4em,
    framextopmargin=10pt,
    framexbottommargin=10pt,
    frame=tb, framerule=0pt,
}

\date{}

\author{
    \scriptsize
    \parbox{0.25\textwidth}{
       \small
       Viola Wiegand \\
       \scriptsize
       Centre for Corpus Research\\ College of Arts and Law\\ University of Birmingham \\
       \texttt{v.wiegand@bham.ac.uk}
    }
    \parbox{0.25\textwidth}{
       \small
       Anna Čermáková \\
       \scriptsize
       Centre for Corpus Research\\ College of Arts and Law\\ University of Birmingham
%       \texttt{@bham.ac.uk}
    }
    \parbox{0.25\textwidth}{
       \small
       Michaela Mahlberg \\
       \scriptsize
       Centre for Corpus Research\\ College of Arts and Law\\ University of Birmingham \\
       \texttt{m.a.mahlberg@bham.ac.uk}
    }
    \\ \\
    \parbox{0.30\textwidth}{
       \small
       Jamie Lentin \\
       \scriptsize
       Shuttle Thread\\ Manchester \\
       \texttt{jamie.lentin@shuttlethread.com}
    }
    \parbox{0.30\textwidth}{
        \small Anthony Hennessey \\
        \scriptsize
        Statistics and Probability\\ School of Mathematical Sciences\\ University of Nottingham \\
        \texttt{anthony.hennessey@nottingham.ac.uk}
    }
}

\title{README}

\begin{document}

\nocite{*}

\maketitle

\tableofcontents

\section{Cleaning}
The sources were the Gutenberg plain text UTF-8 files.

\begin{enumerate}
    \item \label{lst:non_auth} Remove non-authorial text.
    \item \label{lst:title} Reformat the book title and author to make consistent across all texts. 
    \item \label{lst:chapters} Reformat chapter headings to make consistent across all texts. 
    \item \label{lst:line_endings} Convert to unix line endings.
    \item \label{lst:ascii_7} Convert to 7-bit ASCII.
    \item \label{lst:underscores} Delete all underscores. Generally used to typeset emphasis.
\end{enumerate}

Steps \ref{lst:non_auth}, \ref{lst:title} and \ref{lst:chapters} were done manually.
Step \ref{lst:ascii_7} unifies the use of hyphens, apostrophes and quotes across the texts; 7-bit ASCII\footnote{\url{https://tools.ietf.org/html/rfc20}} is a subset of UTF-8\footnote{\url{https://tools.ietf.org/html/rfc3629}} and so the files may be treated as UTF-8.
Texts were converted to ASCII using Version 1.30 of the Perl module \texttt{Text::Unidecode}\footnote{\url{http://search.cpan.org/perldoc?Text::Unidecode}}.

Steps \ref{lst:line_endings}, \ref{lst:ascii_7} and \ref{lst:underscores} can be automated using the simple shell script \texttt{clean.sh}
\lstinputlisting[language=bash]{./clean.sh}


\section{Corpora}

\subsection{ChiLit - Children's Literature}
From work by Anna Čermáková.

\subsubsection{Included texts}
\printbibliography[heading=none,keyword=ChiLit]

\end{document}
