\documentclass[a4paper,10pt]{paper}

\usepackage[utf8]{inputenc}
\setlength{\parindent}{0pt}
%% say no to serif
\renewcommand*\familydefault{\sfdefault}
%% allow arbitrary font sizes
 \usepackage{lmodern}%
\usepackage{amsmath}
\usepackage[inner=2.5cm,outer=2.5cm,bottom=2.5cm,top=2.5cm]{geometry}

%% allow referencing of enumerated lists
\usepackage{enumitem}

%% bibliography
\usepackage{nameref}
\usepackage[british]{babel}
\usepackage{csquotes}
\usepackage[
    style=apa,
    backend=biber,
    url=true, 
    doi=false,
    eprint=false,
    bibencoding=utf8
]{biblatex}
\DeclareLanguageMapping{british}{british-apa}
\DeclareFieldFormat{apacase}{#1}  % otherwise apa sentence cases
\addbibresource{corpora.bib}

% ==========
% non-R code
% ==========

\RequirePackage{listings}
\lstset{
    basicstyle=\footnotesize,
    xleftmargin=1.4em,
    framextopmargin=10pt,
    framexbottommargin=10pt,
    frame=tb, framerule=0pt,
}

\date{}

\author{
    \scriptsize
    \parbox{0.25\textwidth}{
       \small
       Anna Čermáková \\
       \scriptsize
       Centre for Corpus Research\\ College of Arts and Law\\ University of Birmingham\\
       \texttt{a.cermakova@bham.ac.uk}
    }
    \parbox{0.30\textwidth}{
        \small Anthony Hennessey \\
        \scriptsize
        Statistics and Probability\\ School of Mathematical Sciences\\ University of Nottingham \\
        \texttt{anthony.hennessey@nottingham.ac.uk}
    }
    \\ \\
    \parbox{0.30\textwidth}{
       \small
       Jamie Lentin \\
       \scriptsize
       Shuttle Thread\\ Manchester \\
       \texttt{jamie.lentin@shuttlethread.com}
    }
    \parbox{0.25\textwidth}{
       \small
       Michaela Mahlberg \\
       \scriptsize
       Centre for Corpus Research\\ College of Arts and Law\\ University of Birmingham \\
       \texttt{m.a.mahlberg@bham.ac.uk}
    }
    \parbox{0.25\textwidth}{
       \small
       Viola Wiegand \\
       \scriptsize
       Centre for Corpus Research\\ College of Arts and Law\\ University of Birmingham \\
       \texttt{v.wiegand@bham.ac.uk}
    }
}

\title{INDEX}

\begin{document}

\nocite{*}

\maketitle

\tableofcontents

\section{Corpora}

\printbibliography[heading=none,keyword=corpus]

\subsubsection{DNov - Included texts}

\printbibliography[heading=none,keyword=DNov]

\subsubsection{19C - Included texts}

\printbibliography[heading=none,keyword=19C]

\subsubsection{ChiLit - Included texts}

\printbibliography[heading=none,keyword=ChiLit]

\subsubsection{ArTs - Included texts}

\printbibliography[heading=none,keyword=ArTs]

\subsubsection{AAW- Included texts}

\printbibliography[heading=none,keyword=AAW]

\subsection{Updating INDEX.pdf} \label{se:update_index}
The INDEX pdf is compiled with \texttt{latexmk}. On Linux:
\begin{verbatim}
sudo apt install latexmk texlive-latex-extra texlive-bibtex-extra biber
latexmk --outdir=./build --pdf INDEX.tex && cp build/INDEX.pdf .
\end{verbatim}

On Mac, use the TeX Live distribution of your preferred editor, e.g. Texmaker + MacTeX.

\end{document}
